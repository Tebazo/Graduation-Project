\chapter{Anexo A - C�digo Fonte}
\section{startUp.java}
Essa classe � respons�vel por iniciar toda a plataforma de agentes. \\
\lstinputlisting{codigo/startUp.java}
\section{AgenteTarefa.java}
Essa classe representa um Agente Tarefa na sociedade multiagente. \\
\lstinputlisting{codigo/AgenteTarefa.java} 
\section{AgenteEscalonador.java}
Essa classe representa o Agente Escalonador da sociedade multiagente. \\
\lstinputlisting{codigo/AgenteEscalonador.java}
\section{AgenteChefe.java}
Essa classe representa o Agente Chefe da sociedade multiagente. \\
\lstinputlisting{codigo/AgenteChefe.java}
\section{ProcessaRelacoes.java}
Essa classe auxilia na manipula��o de estruturas que representam o grafo de rela��es entre os agentes. \\
\lstinputlisting{codigo/ProcessaRelacoes.java}
\section{OutputUtils.java}
Essa classe padroniza todo o output dos agentes para a sa�da padr�o. \\
\lstinputlisting{codigo/OutputUtils.java}
\section{Tarefa.java}
Essa classe representa uma tarefa no sistema. \\
\lstinputlisting{codigo/Tarefa.java} 
\section{Estados.java}
Essa classe representa os estados que um agente pode apresentar. \\
\lstinputlisting{codigo/Estados.java}
\section{Habilidades.java}
Essa classe armazena todas as poss�veis habilidades dos agentes.  \\
\lstinputlisting{codigo/Habilidades.java}