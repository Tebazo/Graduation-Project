\chapter{Resumo}

Embora redes complexas seja um tema atual, sua história tem inicio no começo do século XX. Primeiramente somente os envolvidos com as ciências exatas, especificamente os físicos e matemáticos, tinham interesse nessa área,como é o caso do matemático Ëuler, que criou a primeira teoria dos grafos.Devido essa ideia, diversos estudiosos de diferentes áreas motivaram-se a compreender como era a construção desses grafos e suas diversas propriedades. Essa forma de percepção dos elementos como redes, seria crucial para a compreensão das relações complexas do mundo contemporâneo.

          Este projeto de iniciação científica tem como objetivo geral investigar questões relacionadas à análise de dados em sistemas complexos. Mais especificamente na dinâmica de redes sociais, tendo como base principal a teoria das Redes Sociais. 
          
          
         Após estudar e aplicar modelos utilizados na extração de dados, haverá analises dos resultados comparando os modelos. Além de propor sugestões para otimizar os modelos utilizados.
