%%ARQUIVO PARA CONFIGURACAO DAS PAGINAS


\usepackage[brazilian]{babel} %Package para identifica��o de portugues - brasileiro
\usepackage[T1]{fontenc} % utiliza��o na separa��o de s�labas
\usepackage{ae} %auxilio na separa��o de s�labas
%\usepackage{natbib} %para auxiliar nas referencias, caso necess�rio
\usepackage{cite}
\usepackage{amsmath}
\usepackage{float}
\usepackage{wrapfigg}
\usepackage[ansinew]{inputenc}
\usepackage{graphics}
\usepackage{calc}
\usepackage{graphicx,color}
\usepackage{amsthm,amsfonts}
\usepackage{amssymb}
\usepackage{epsfig}
%\usepackage[center]{caption}
%\usepackage[font=it,center]{caption}%font=it/textfont=it
\usepackage[top=3cm,left=3cm,right=2cm,bottom=2cm]{geometry} %dimensoes da pagina
\usepackage{url}
\usepackage{rotating}
%%compatibilidade com o estilo MEMOIR e package Fancyhdr
\let\footruleskip\relax % for compatibility of memoir and fancyhdr
\let\rm\rmfamily        % for compatibility of memoir and blindtext (demo only)
\usepackage{fancyhdr}
\usepackage{titlesec}
\usepackage{xcolor}
\usepackage{eso-pic}
\usepackage{ifthen} %comando para gerar if`s else


\usepackage{listings}
\usepackage{color}
\usepackage{textcomp}
\definecolor{listinggray}{gray}{0.9}
\definecolor{lbcolor}{rgb}{0.9,0.9,0.9}
\lstset{
	backgroundcolor=\color{white},
	tabsize=4,
	rulecolor=,
	language=java,
        basicstyle=\scriptsize,
        upquote=true,
        aboveskip={1.5\baselineskip},
        columns=fixed,
        showstringspaces=false,
        extendedchars=true,
        breaklines=true,
        prebreak = \raisebox{0ex}[0ex][0ex]{\ensuremath{\hookleftarrow}},
        frame=single,
        showtabs=false,
        showspaces=false,
        showstringspaces=false,
        identifierstyle=\ttfamily,
        keywordstyle=\color[rgb]{0,0,1},
        commentstyle=\color[rgb]{0.133,0.545,0.133},
        stringstyle=\color[rgb]{0.627,0.126,0.941},
}


%%%%%%%%%% definicao de header das paginas %%%%%%%%%%%%%%%%

\pagestyle{fancy}                         %Forces the page to use the fancy template
\renewcommand{\chaptermark}[1]{\markboth{\textbf{\thechapter}\ \emph{#1}}{}}
\renewcommand{\sectionmark}[1]{\markright{\thesection.\ #1}}

\fancyhf{}                                %Clears all header and footer fields, in preparation.
\fancyhead[LE,RO]{\textbf{\thepage}}      %Displays the page number in bold in the header, to the left on even pages and
                                          %to the right on odd pages.
\fancyhead[RE]{\nouppercase{\leftmark}}   %Displays the upper-level (section) information - as determined above - in non-
                                          %upper case in the header, to the left on odd pages.
\fancyhead[LO]{\rightmark}                %Displays the lower-level (chapter) information - as determined above - in the 							 %header, to the left on odd pages.
\renewcommand{\headrulewidth}{0.5pt}      %Underlines the header. (Set to 0pt if not required).
%\renewcommand{\footrulewidth}{0.5pt}      %Underlines the footer. (Set to 0pt if not required).

%%%%%%%%%% fim da definicao de header das paginas %%%%%%%%%%%%%%%%


%Para gera��o do arquivo com links, caso n�o queira � s� comentar as duas  linhas abaixo
\usepackage{hyperref}
\hypersetup{colorlinks, debug=false,  linkcolor=black,  citecolor=black, urlcolor=blue, pdftitle={Um Estudo sobre a Perspectiva da Modelagem de Sistemas Multiagentes via a Teoria das Redes Sociais}, pdfauthor={Andr� Filipe de Moraes Batista}}

%%%%%%%%%%%%%%%%%%%%%%%%%%%%%%%%%%%%%%%%%%%%%%%%%%%%%%%%%%%%%%%%%%%%%%%

%%GERACAO DE DRAFT

\ifdraftdoc
%\usepackage{draftwatermark}
\usepackage[firstpage]{draftwatermark}
%\usepackage{watermark}

% Use the following to make modification
\SetWatermarkAngle{0}
\SetWatermarkLightness{0.6}
\SetWatermarkFontSize{3cm}
\SetWatermarkScale{1}
\SetWatermarkText{DRAFT}

\else
%%do nothing
\fi

%%%%%%%%%%%%%%%%%%%%%%%%%%%%%%%%%%%%%%%%%%%%%%%%%%%%%%%%%%%%%%%%%%%%%%%

\setcounter{secnumdepth}{10}
\setcounter{tocdepth}{10}

%%%%%%%%%%%%%%%%%%%%%%%%%%%%%%%%%%%%%%%%%%%%%%%%%%%%%%%%%%%%%%%%%%%%%%%

%COMANDOS CRIADOS

\newcommand{\media}[1] {
\langle{#1}\rangle
}

%comando \hoje
%nada mais eh do que o comando \today com meses com letras maiusculas

\def\hoje{\number\day\space de\space\ifcase\month\or
 Janeiro\or Fevereiro\or Mar\c{c}o\or Abril\or Maio\or Junho\or
      Julho\or Agosto\or Setembro\or Outubro\or Novembro\or Dezembro\fi
\space de\space\number\year}


\newcommand{\epigrafe}[2]{

{\large \parbox{2cm}{\hspace*{7cm}}}%
\parbox[t]{12cm}
{
\textit{#1}
\par ----- #2.\\\\
}}

\newcommand{\apresentacao}[1] {
\hspace{.2\textwidth} % posicionando a minipage
    \begin{minipage}{.7\textwidth}
          \begin{flushright}
          \begin{sloppypar}
        {{\large \parbox{2cm}{\hspace*{7cm}}}%
\parbox[t]{8.5cm}
{
#1 \\
\par
}}\\[0.3cm]
\end{sloppypar}
\end{flushright}
\end{minipage}
}






%%%%%%%%%%%%%%%%%%%%%%%%%%%%%%%%%%%%%%%%%%%%%%%%%%%%%%%%%%%%%%%%%%%%%%%%%%
%%definicao de novo estilo de capitulo

\usepackage{color,calc,graphicx,soul,fourier} %packages necessarias


%%%ESTA � A COR VERMELHA DEFINIDA, PARA MUDAR A COR � S� MUDAR O RGB ABAIXO.
\definecolor{nicered}{rgb}{.647,.129,.149} %crio uma cor


\makeatletter
\newlength\dlf@normtxtw
\setlength\dlf@normtxtw{\textwidth}
\def\myhelvetfont{\def\sfdefault{mdput}}
\newsavebox{\feline@chapter}
\newcommand\feline@chapter@marker[1][4cm]{%
\sbox\feline@chapter{%
\resizebox{!}{#1}{\fboxsep=1pt%
\colorbox{nicered}{\color{white}\bfseries\sffamily\thechapter}%
}}%
\rotatebox{90}{%
\resizebox{%
\heightof{\usebox{\feline@chapter}}+\depthof{\usebox{\feline@chapter}}}%
{!}{\scshape\so\@chapapp}}\quad%
\raisebox{\depthof{\usebox{\feline@chapter}}}{\usebox{\feline@chapter}}%
}
\newcommand\feline@chm[1][4cm]{%
\sbox\feline@chapter{\feline@chapter@marker[#1]}%
\makebox[0pt][l]{%aka\rlap
\makebox[1cm][r]{\usebox\feline@chapter}%
}}


\makechapterstyle{daleif1}{
\renewcommand\chapnamefont{\normalfont\Large\scshape\raggedleft\so}
\renewcommand\chaptitlefont{\normalfont\huge\bfseries\scshape\color{nicered}}
\renewcommand\chapternamenum{}
\renewcommand\printchaptername{}
\renewcommand\printchapternum{\null\hfill\feline@chm[2.5cm]\par}
\renewcommand\afterchapternum{\par\vskip\midchapskip}
\renewcommand\printchaptertitle[1]{\chaptitlefont\raggedleft##1\par}
}



\makeatother
\chapterstyle{daleif1}  %%aqui eu falo o estilo que vou falar

%%%%--> a seguir defino o estilo do titulo das se��es

%%%Agradecimentos: Ronaldo C. Prati.
%%%Requer: \usepackage{titlesec} \usepackage{xcolor}

  \titleformat{\section} {} {\textcolor{nicered}{\LARGE\bfseries\thesection}} {1em}
    {\LARGE\bfseries\textcolor{nicered}}

     \titleformat{\subsection} {} {\textcolor{nicered}{\Large\bfseries\thesubsection}} {1em}
    {\Large\bfseries\textcolor{nicered}}

         \titleformat{\subsubsection} {} {\textcolor{nicered}{\large\bfseries\thesubsubsection}} {1em}
    {\large\bfseries\textcolor{nicered}}

       \titleformat{\paragraph} {} {\textcolor{nicered}{\large\bfseries\theparagraph}} {1em}
    {\large\bfseries\textcolor{nicered}}

    \let\subsubsubsection=\paragraph
    \setcounter{tocdepth}{4}

%% \titleformat{\part}{\normalfont\normalsize\bfseries}{\thepart}{1em}{}

\newcommand\BackgroundPic{
\put(0,0){
\parbox[b][\paperheight]{\paperwidth}{%
\vfill
\centering
\includegraphics[width=\paperwidth,height=\paperheight,
keepaspectratio]{imagens/fundo.pdf}%
\vfill
}}}


%%modo antigo, mas funcional: (deixei como exemplo para part)


\titleformat{\part} {} {\pagestyle{empty}\textcolor{nicered}{\HUGE\bfseries\thepart}} {1em}
    {\AddToShipoutPicture*{\BackgroundPic}
\pagestyle{empty}\HUGE\bfseries\textcolor{nicered}\ClearShipoutPicture\textcolor{nicered}
}


\definecolor{MyGray}{rgb}{0.82,0.87,1.00}

\newcommand\MyQuoteFrameCommand[1]{%
  \setlength\fboxsep{2ex}\setlength\fboxrule{.5pt}%
   \fcolorbox{black}{MyGray}{%
    \hskip-\fboxsep\hskip-\fboxrule\hskip\leftmargini
     #1%
    \hskip-\fboxsep\hskip-\fboxrule\hskip\leftmargini
   }%

}

\newenvironment{definicao}{%
 \def\FrameCommand{\MyQuoteFrameCommand}%
  \framed  \setlength\parindent{0pt}}%
 {\endframed}


\newenvironment{exercise}%          environment name
{\textbf{Exercise}\begin{itshape}}% begin code
{\end{itshape}}%                    end code



%%%%%%%%%%%%%%%%%%%%%%%%%%%%%%%%%%%%%%%%%%%%%%%%%%%%%%%%%%%%%%%%%%%%%%%%%%%%%
%%%%%%%%%%%%%%%%%%%%%%%%%%%%%%%%%%%%%%%%%%%%%%%%%%%%%%%%%%%%%%%%%%%%%%%%%%%%%
%%%%%%%%%%%%%%%%%%%%%%%%%%%%%%%%%%%%%%%%%%%%%%%%%%%%%%%%%%%%%%%%%%%%%%%%%%%%%
%%%%%%%%%%%%%%%%%%%%%%%%%%%%%%%%%%%%%%%%%%%%%%%%%%%%%%%%%%%%%%%%%%%%%%%%%%%%%
%%%%%%%%%%%%%%%%%%%%%%%%%%%%%%%%%%%%%%%%%%%%%%%%%%%%%%%%%%%%%%%%%%%%%%%%%%%%%
